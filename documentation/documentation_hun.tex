\documentclass{article}
\usepackage{graphicx}
\usepackage[numbers]{natbib}

\begin{titlepage}
    \title{Járművek mozgásának előrejelzése gépi tanulási módszerrel}
    \author{Peter Bence}
    \date{2023 Februar}
\end{titlepage}


\begin{document}
    \bibliographystyle{plainnat}
    \maketitle
    \begin{abstract}
        Járművek mozgásának előrejelzése kereszteződésekben egy komplex és fontos feladat gépi látás 
        és gépi tanulás téren. Matematikai, fizikai, statisztikai és neurális háló modellekkel való 
        megoldások születtek erre a problémára.\cite{1238912}\cite{6696982}\cite{8186191}\cite{8317943} 
        Önvezetés téren is sokan kutatnak e probléma megoldása miatt. 
        Én egy kicsit más szemszögből közelítettem meg ezt a problémát. Madártávlatból felvett felvételekből
        nyertem ki járművek trajektóriájit YOLO konvolúciós neurális háló\cite{wang2022yolov7} és 
        DeepSORT\cite{Wojke2018deep}\cite{Wojke2017simple} objektum követő algoritmus segítségével.
        Klaszterezéssel felfedtem a szabályos-ságokat, majd ezek szabályosságok alapján klasszifikációs
        modelleket taní-tottam be.
    \end{abstract}
    \tableofcontents
    \bibliography{biblio}
\end{document}