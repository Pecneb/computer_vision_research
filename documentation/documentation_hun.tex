\documentclass{article}
\usepackage{graphicx}
\usepackage[numbers]{natbib}
\usepackage{float}

\begin{titlepage}
    \title{Járművek mozgásának előrejelzése gépi tanulási módszerrel}
    \author{Péter Bence\\Mérnökinformatika BSc}
    \date{2023 Február}
\end{titlepage}


\begin{document}
    \bibliographystyle{plainnat}
    \maketitle
    \begin{figure}[H]\centering
        \includegraphics[width=1\columnwidth]{sze_givk_logo.png}
    \end{figure}
    \newpage
    \begin{abstract}
        Ebben a kutatásban gépi tanulási módszerrel törekszünk megoldani
        járművek trajektóriájának előrejelzését komplex forgalmi helyzetekben.
        Madártávlatból felvett felvételekből nyertünk ki járművek trajektóriájit 
        YOLO\cite{wang2022yolov7} konvolúciós neurális háló és DeepSORT\cite{Wojke2018deep} 
        objektum követő algoritmus segítségével. Különböző klaszterezése algoritmusokat 
        tesztel-tünk, többfajta tulajdonság vektorral, hogy minél pontosabban kategorizálni 
        tudjuk a járművek mozgását. Ezek kategóriák felhasználásával klasszifikációs modelleket 
        tanítottunk be, több fajta tulajdonság vektorral, amiknek pontosságát megmértük 
        és összehasonlítottuk. Kutatásunk során a teljes tanítási és tesztelési folyamat 
        gyorsítására és automatizálásá-ra nagy hangsúlyt fektettünk, ennek megoldására egy 
        keretrendszert fejlesztettünk ki. A kutatásból azt a tanulságot tudjuk leszűrni,
        hogy a klaszterezési folyamatot nehéz automatizálni, mivel a klaszterezéshez
        használt modellek hiperparamétereit nem lehet univerzálisan minden ke-reszteződésre 
        felhasználni. A klasszifikációhoz használt modellek nem csak pontosnak,
        hanem gyorsnak is kell lennie, ha valós időben akarjuk használni.
        A detektálásnál felhasznált DeepSORT\cite{Wojke2018deep} objektum követő algoritmus
        hiperparamétereit is gondosan meg kell választani, hogy ne legyenek anomáliák a tanító
        adathalmazban.
    \end{abstract}
    \tableofcontents
    \bibliography{biblio}
\end{document}