\documentclass{article}
\usepackage{graphicx}
\usepackage[numbers]{natbib}
\usepackage{float}
\usepackage{hyperref}
\usepackage{comment}
\usepackage{ragged2e}


\begin{titlepage}
    \title{Járművek trajektóriájinak előrejelzése machine learning módszerrel}
    \author{Péter Bence\\Mérnökinformatika BSc 6. félév\\Konzulensek:\\Dr. Horváth András egyetemi docens\\Agg Áron PhD hallgató}
    \date{2023 Február}
\end{titlepage}

\begin{document}
    \bibliographystyle{plainnat}
    \begin{titlepage}
        \maketitle
        \begin{figure}[H]\centering
            \includegraphics[width=1\columnwidth]{sze_givk_logo.png}
        \end{figure}
    \end{titlepage}
    \begin{abstract}
        Az ITS (intelligent transportation system) egyre nagyobb teret 
        hódít napjainkban és rengeteg különböző területen alkalmazzák 
        ezeket a rendszereket. A közlekedési csomópontok elemzése egy 
        frekventált terület az ITS alkalmazásában.
        Célunk, gépi látás és gépi tanulás felhasználásával, közlekedési
        csomópontok elemzésének automatizálása és felgyorsítása. A kutatásban
        lefektetett alapgondolatokat, kifejlesztett keretrendszert és 
        a felmerülő probélmák megoldásait, a gyakorlatban balesetek megelőzésére,
        renitens viselkedések kiszűrésére és forgalomirányító renszerek támogatásá-ra
        lehet használni.  
        A kutatásban egy trajektória osztályozó módszert ismertetünk, amely objektumdetektálás 
        és objektumkövetés segítségével elemezi a közlekedési 
        csomópontokban elhaladó járművek mozgását. A mozgásuk alapján 
        automatikusan klaszterezi a trajektóriákat, majd gépi tanulás 
        segítségével predikciót ad az újonnan belépő járművek kilépési 
        pontjára. A módszerhez 5 különböző közlekedési csomópontban 
        készített saját videó adatbázisunkat használtuk fel.
        A tesztelt klaszterezési mód-szerek közül (OPTICS, BIRCH, KMeans, DBSCAN)
        az OPTICS algoritmus bizonyult legjobbnak trajektórák klaszterezésére.
        Összehasonlí-tottunk több különböző klasszifikációs módszert 
        a legpontosabb predikció eléréséhez, amelyek: KNN, SVM, GP, DT, 
        GNB, MLP, SGD. A tanul-mányban bemutatott eljárások közül az 
        SVM adta a legpontosabb 90\%-os eredményt.
        %Ezt a pontosságot valós idejű futás közben 5 fps mellett érte el.
        %Ebből azt a következtetést lehet levonni, hogy jobb sebesség elérése
        %érdekében, vagy a feature vectorok dimenzióját kell csökkenteni, vagy
        %érdemes neurális hálót alkalmazni a klasszifikációhoz.
        %A forráskód megtalálható ebben a github repositoriban 
        %\url{https://www.github.com/Pecneb/computer_vision_research}
    \end{abstract}
    %\tableofcontents

    %\bibliography{biblio}
    %\begin{CCSXML}
    %    <ccs2012>
    %        <concept>
    %            <concept_id>10010147.10010257.10010293</concept_id>
    %            <concept_desc>Computing methodologies~Machine learning approaches</concept_desc>
    %            <concept_significance>500</concept_significance>
    %            </concept>
    %    </ccs2012>
    %\end{CCSXML}
    %\ccsdesc[500]{Computing methodologies~Machine learning approaches}
\end{document}