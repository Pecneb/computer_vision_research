\documentclass[acmtog, authorversion]{acmart}
\usepackage{graphicx}
%\usepackage[numbers]{natbib}
%\usepackage{float}
%\usepackage{hyperref}
%\usepackage{comment}
%\usepackage{ragged2e}
\setcopyright{none}
\citestyle{acmauthoryear}


\begin{document}

\title{Járművek trajektóriájinak előrejelzése machine learning modellekkel}

\author{Péter Bence Mérnökinformatika BSc 6. félév}%\\Mérnökinformatika BSc 6. félév\\Konzulensek:\\Dr. Horváth András egyetemi docens\\Agg Áron PhD hallgató}
\authornotemark[1]
\affiliation{
    \institution{Széchenyi István Egyetem}
    \city{Győr}
    \country{Hungary}}

\author{Dr. Horváth András}
\affiliation{
    \institution{Széchenyi István Egyetem}
    \city{Győr}
    \country{Hungary}}

\author{Agg Áron PhD hallgató}
\affiliation{
    \institution{Széchenyi István Egyetem}
    \city{Győr}
    \country{Hungary}}

\begin{teaserfigure}
\includegraphics[width=1\columnwidth]{sze_givk_logo.png}
\Description{egyetem_kar}
\end{teaserfigure}

\begin{abstract}
    Az ITS (intelligent transportation system) egyre nagyobb teret 
    hódít napjainkban és rengeteg különböző területen alkalmazzák 
    ezeket a rendszereket. A közlekedési csomópontok elemzése egy 
    frekventált terület az ITS alkalmazásában.
    Célunk, gépi látás és gépi tanulás felhasználásával, közlekedési
    csomópontok elemzésének automatizálása és felgyorsítása. A kutatásban
    lefektetett alapgondolatokat, kifejlesztett keretrendszert és 
    a felmerülő probélmák megoldásait, a gyakorlatban balesetek megelőzésére,
    renitens viselkedések kiszűrésére és forgalomirányító renszerek támogatásá-ra
    lehet használni.  
    A kutatásban egy trajektória osztályozó módszert ismertetünk, amely objektumdetektálás 
    és objektumkövetés segítségével elemezi a közlekedési 
    csomópontokban elhaladó járművek mozgását. A mozgásuk alapján 
    automatikusan klaszterezi a trajektóriákat, majd gépi tanulás 
    segítségével predikciót ad az újonnan belépő járművek kilépési 
    pontjára. A módszerhez 5 különböző közlekedési csomópontban 
    készített saját videó adatbázisunkat használtuk fel.
    A tesztelt klaszterezési mód-szerek közül (OPTICS, BIRCH, KMeans, DBSCAN)
    az OPTICS algoritmus bizonyult legjobbnak trajektórák klaszterezésére.
    Összehasonlí-tottunk több különböző klasszifikációs módszert 
    a legpontosabb predikció eléréséhez, amelyek: KNN, SVM, GP, DT, 
    GNB, MLP, SGD. A tanul-mányban bemutatott eljárások közül az 
    SVM adta a legpontosabb 90\%-os eredményt.
    %Ezt a pontosságot valós idejű futás közben 5 fps mellett érte el.
    %Ebből azt a következtetést lehet levonni, hogy jobb sebesség elérése
    %érdekében, vagy a feature vectorok dimenzióját kell csökkenteni, vagy
    %érdemes neurális hálót alkalmazni a klasszifikációhoz.
    %A forráskód megtalálható ebben a github repositoriban 
    %\url{https://www.github.com/Pecneb/computer_vision_research}
\end{abstract}

\maketitle

\tableofcontents

\section{Bevezetés}
A városok növekedése egyre nagyobb forgalomhoz vezet, ami a balesetek, forgalmi dugók számát növeli és a levegő minősége is romlik.
Az ITS (intelligent transportation system) fejlesztése a városokban erre megoldást jelenthet. Ez magába foglalja az információs és
kommunikációs technológiák, mint pélául szenzorok, kamerák, kommunikációs hálózatok és adat elemzés fejlesztését. 5G hálózatokon
keresztül, ezek a technológiák összeköthetők a közlekedési eszközökkel. Ehhez okos forgalomirányítási rendszerek kifejlesztésére
van szükség, amik információval tudnak szolgáni a járművekbe szerelt informatikai rendszereknek.
A legértékesebb információt a közlekedésben részvevő járművek jelen és jövőbeli pozíciója jelenti. Pontos és gyors trajektória 
előrejelző rendszerek kifejlesztése egy nagy kihívás és egyre növekszik irántuk a kereslet. E kutatási terület kiforratlanságából
eredően, nem lehet csak úgy belevágni, és egyből machine learning modelleket vagy neurális hálókat tanítani. Tanító adatok gyűjtése, 
és mérőszámok kifejlesztése (amivel a tesztelni kívánt modellek pontosságát tudjuk mérni) is a kutatáshoz tartoznak.  
Ebben a kutatásban erre a problémára törekszünk egy módszertant és keretrendszert kifejleszteni, emellett klaszterezési és klasszifikációs
algoritmusokat tesztelni. A tanító adatok előállításához, objektumok detektálására a YOLOv7 \cite{wang2022yolov7}
konvolúciós neurális hálót használtuk, ez a konvolúciósl neurális háló architektúra nem csak nagy pontosságot hanem sebességet is nyújt nekünk. 
Emellett képkockáról képkockára követni is kell tudni a detektált objektumokat. Erre is sok megoldás található manapság, erre a feladatra
a DeepSORT \cite{Wojke2018deep} nevezetű algoritmust használtuk, ez kálmán filtert és konvolúciós neurális hálót használ az objektumok követésére.



\bibliographystyle{ACM-Reference-Format}
\bibliography{biblio}

\begin{CCSXML}
    <ccs2012>
        <concept>
            <concept_id>10010147.10010257.10010293</concept_id>
            <concept_desc>Computing methodologies~Machine learning approaches</concept_desc>
            <concept_significance>500</concept_significance>
            </concept>
    </ccs2012>
\end{CCSXML}
\ccsdesc[500]{Computing methodologies~Machine learning approaches}
\end{document}